	Ceci contient des liens pour 7 catégories, toutes liées à la science, et que je trouve intéressantes. Je souhaite faire remarquer qu'en aucun cas je fus rémunéré sous quelque forme que ce soit pour ajouter des liens à la liste ci-desous ! Vous pouvez également trouver mes applis préférées sur mon blog (http://www.sciences.ch/blog/). 
	
	\begin{itemize}	 
		\item[$-$] {\Large \ding{52}} Sites web de grande qualité (design & contenu)
		\item[$-$] {\Large \ding{45}} Contenu avec développements et démonstrations
		\item[$-$] {\Large \ding{41}} Forums web
		\item[$-$] {\Large \ding{36}} Logiciels Scientifiques (commerciaux, sharewares, freewares) à télécharger
		\item[$-$] {\Large \ding{229}} Livres, publications, magazines, articles (à voir ou télécharger)
		\item[$-$] {\Large \ding{44}} Sites Cool/divertissants
		\item[$-$] {\Large \ding{170}} Mes sites favoris
		\item[$-$] {\Large \ding{73}} A voir absolument
	\end{itemize}
	
	S'il y avait 3 liens à mettre en exergue, ce serait \href{http://www.google.com}{\color{blue} Google}, \href{http://www.wikipedia.com}{\color{blue} Wikipedia} et \href{http://www.youtube.com}{\color{blue} YouTube} à qui nous devons beaucoup !		

	\pagebreak

	\section{Exact Sciences}

	{\Large \ding{52}}{\Large \ding{45}}{\Large \ding{36}}{\Large \ding{44}}{\Large \ding{170}}{\Large \ding{73}}\bcdfrance{} ChronoMath, petite histoire des mathematiques, est un site éducatif continuellement remis à jour par Serge Mehl depuis 1988, en tant que professeur de maths en Afrique durant de nombreuses années. Plus de 450 mathématiciens (et leur travail) y sont passés en revue...! A voir !\\
	\href{http://www.chronomath.com}{\color{blue} http://www.chronomath.com}
	
	{\Large \ding{41}}{\Large \ding{36}}{\Large \ding{229}}\bcdfrance{} Ce site diffuse l'actualité des mathématiques news, une encyclopédie incluant dictionaire, biographies et formules, ainsi que des fichiers sur divers sujets mathématiques et un forum de discussion. Ce site est particulièrement impressionnant au vue de la quantité d'information disponible en téléchargement. \\
	\href{http://www.bibmath.net}{\color{blue} http://www.bibmath.net} 
	
	{\Large \ding{52}}{\Large \ding{45}}\bcdfrance{} Un bon site contenant du matériel pédagogique en physique (synophysique). Le design pourrait être revu (la navigation par frames)... mais la qualité de son contenu en fait un site incontournable.\\
	\href{http://www.sciences.univ-nantes.fr/physique/perso/blanquet/frame3.htm}{\color{blue} http://www.sciences.univ-nantes.fr/physique/perso/blanquet/frame3.htm}
	
	{\Large \ding{52}}{\Large \ding{45}}{\Large \ding{36}}\bcdfrance{} beaucoup de PDFs sur l'algebre, la géométrie et l'analyse.\\
	\href{http://c.caignaert.free.fr}{\color{blue} http://c.caignaert.free.fr}
	
	{\Large \ding{52}}{\Large \ding{45}}{\Large \ding{36}} arXiv est un site d'archivage pour les documents scientifiques électroniques publiés. Le temps écoulé entre la réalisation d'un projet et la publication du travail du chercheur peut aller jusqu'à une année. A l'échelle de la recherche, c'est une longue durée. La mise en place d'arXiv eu pour but de résoudre ce soucis de durée et de coût de publication.\\
	\href{http://arxiv.org}{\color{blue} http://arxiv.org}
	
	{\Large \ding{52}}{\Large \ding{45}}{\Large \ding{36}}\bcdfrance{} Le "HAL archives ouvertes" est une plateforme multi-disciplinaire où sont rassemblées des documents de recherche, des publications scientifiques produites par des universités, institutions de recherches, laboratoires publics ou privés, de sources françaises ou étrangères. Il y a même des ouvrages scientifiques publiés récements par des maisons d'éditions françaises.\\
	\href{https://hal.archives-ouvertes.fr/}{\color{blue} https://hal.archives-ouvertes.fr/}
	
	\pagebreak
	\section{Publications/Magazines}

	{\Large \ding{52}}{\Large \ding{41}}{\Large \ding{229}}{\Large \ding{170}}{\Large \ding{73}}\bcdfrance{} Nous pourrons considérer ce prochain site comme l'équivalent du précédent, mais pour les ingénieurs français. Cependant, il est qualitativement meilleur, avec la même quantité d'information, mais plus homogène. Le seul regret sera peut-être que l'accès à certains éléments n'est pas intuitif au début, et parfois payant.\\
	\href{http://www.techniques-ingenieur.fr}{\color{blue} http://www.techniques-ingenieur.fr}
	
	{\Large \ding{52}}{\Large \ding{229}}{\Large \ding{170}}{\Large \ding{73}}\bcdfrance{} Absolument excellent et à voir sans hésiter ! De nombreux cours complets de l'École Polytechnique (France) y sont publiés et disponibles en téléchargement gratuitement au format PDF (est-ce que ça va durer ?). Environ 1,000 documents éducatifs sont disponibles, avec une qualité variable, mais la pertinence veut bien la rareté des sujets.\\
	\href{https://moodle.polytechnique.fr/}{\color{blue} https://moodle.polytechnique.fr/}
	
	{\Large \ding{52}}{\Large \ding{229}}{\Large \ding{170}}{\Large \ding{73}}\bcdfrance{}  Le site en lui même n'est pas extra (c'est un peu désolant) mais le magazine "Tangente" auquel il propose un abonnement (payant) est très bien pour les aficionados des mathématiques et de l'actualité des mathématiques.\\
	\href{http://tangente.poleditions.com/}{\color{blue}http://tangente.poleditions.com/}
	
	{\Large \ding{52}}{\Large \ding{229}}{\Large \ding{170}}{\Large \ding{73}}\bcdfrance{} Un magazine de la même famille que "Tangente" mais avec un niveau technique et académique bien plus élevé (selon moi). Le contenu est particulièrement orienté vers les mathématiques pures sans lien direct ou explicite avec leur application physique, économique ou d'ingénieurie.\\
	\href{http://www.quadrature.info}{\color{blue}http://www.quadrature.info}
	
	{\Large \ding{52}}{\Large \ding{229}}{\Large \ding{170}}{\Large \ding{73}} Le site en lui même n'est aussi pas génial (et c'est encore dommage) mais les informations proposées sont historiques ! (Le lien pointe sur la partie européenne du site, à moins que vous ne lisiez le русский язык)\\
	\href{http://urss.ru/cgi-bin/db.pl?lang=en&blang=en&page=Catalog&list=1}{\color{blue}http://urss.ru}
	
	{\Large \ding{52}}{\Large \ding{229}}{\Large \ding{170}}{\Large \ding{73}}\bcdfrance{} Excellent site web avec une énorme quantité de ressources (annuaire de liens) vers de la documentation électronique en lien avec les mathématiques, en français et anglais seulement. Ce site regroupe des liens vers les pages des auteurs des documents (cela ne demande pas un serveur web énorme...).\\
	\href{http://mathslinker.chez-alice.fr}{\color{blue}http://mathslinker.chez-alice.fr}
	
	{\Large \ding{229}}\bcdfrance{} Ici, nous avons une librairie en ligne, proposant les publications en français des plus grand mathématiciens du XIXè et XXè siècles. A voir !\\
	\href{http://matwbn.icm.edu.pl/wyszukiwarka.php}{\color{blue}http://matwbn.icm.edu.pl/wyszukiwarka.php}
	
	{\Large \ding{52}}{\Large \ding{229}}{\Large \ding{170}}{\Large \ding{73}}\bcdfrance{} Maison d'édition vendant des ouvrages fondamentaux sur les mathématiques, physique, histoire et philosophie des sciences.\\
	\href{http://www.gabay-editeur.com/}{\color{blue}http://www.gabay-editeur.com/}
	
	\pagebreak
	{\Large \ding{52}}{\Large \ding{229}}{\Large \ding{170}}{\Large \ding{73}}\bcdfrance{} Les éditions Eyrolles - un excellent site regroupant bon nombre de travaux en anglais et français. On y trouve des livres de divers auteurs, par exemple Wiley, Springer, etc. Cela vaut le coup d'aller jeter un oeil dans les sections "Mathematiques" et "Physique" où il y a de bonnes choses... \\
	\href{http://www.eyrolles.com}{\color{blue}http://www.eyrolles.com}
	
	{\Large \ding{52}}{\Large \ding{229}}{\Large \ding{170}}{\Large \ding{73}}\bcdfrance{} Dunod Editions - litérature scientifique contemporaine (universitaire et post-doc). Les livres de cette maison d'édition sont simplement excellents ! Les developpements sont très détaillés, une référence technique assurée, sans être une référence pédagogique.\\
	\href{http://www.dunod.com}{\color{blue}http://www.dunod.com}
	
	{\Large \ding{229}}{\Large \ding{170}} les éditions Springer - Litérature scientifique niveau doctorat. La navigation n'est pas évidente, les menus sont arrangés n'importe comment, mais le contenu est de haut.... très haut niveau.\\
	\href{http://www.springer.com/fr/}{\color{blue}http://www.springer.com/fr/}
	
	{\Large \ding{229}}{\Large \ding{170}}{\Large \ding{73}}\bcdfrance{} Le projet NUMDAM, conduit par MathDoc (UMS 5638 CNRS - UJF) délégation du CNRS. C'est la bibliothèque numérique française de mathématiques.\\
	\href{http://www.numdam.org}{\color{blue}http://www.numdam.org}
	
	{\Large \ding{52}}{\Large \ding{229}}{\Large \ding{170}}{\Large \ding{73}} Wrox Editions propose des ouvrages de programmation informatique (niveau expert). A mon avis (autant que je sache) cette maison d'édition est une référence mondiale en matière d'ouvrages sur les langages de programmation.\\
	\href{http://www.wrox.com}{\color{blue}http://www.wrox.com}
	
	{\Large \ding{52}}{\Large \ding{170}}{\Large \ding{73}} Les éditions de presse de l'université de Cambridge proposent également de la litérature scientifique niveau post-doctorant. Quelque part un équivalent des éditions Springer avec un site mieux structuré. Une inscription gratuite vous permet de recevoir leur catalogue.\\
	\href{http://www.cambridge.org}{\color{blue}http://www.cambridge.org}
	
	
	{\Large \ding{45}}{\Large \ding{229}}{\Large \ding{170}}{\Large \ding{73}} La bibliothèque nationale de France (BNF) offre gratuitement de télécharger des documents scientifiques dont le copyright a plus de 75 ans. Le téléchargement n'est pas très intuitif, mais l'on y trouve de très intéressants documents.\\
	\href{http://gallica.bnf.fr}{\color{blue} http://gallica.bnf.fr}
	
	{\Large \ding{45}}{\Large \ding{229}}{\Large \ding{170}}{\Large \ding{73}} Le moteur de recherche de la "Physical Review Letters" (PRL) permet, moyennant abonnement payant, un accès access aux articles de recherche scientifique en physique théorique et expérimentale sur les 100 dernières années. Le niveau de détail des articles n'est pas très élevé, mais ce n'est pas l'objectif principal pour les spécialistes.\\
	\href{http://prola.aps.org/search}{\color{blue} http://prola.aps.org/search}
	
	\pagebreak
	{\Large \ding{45}}{\Large \ding{229}}{\Large \ding{73}} Editions Lavoisier - De très bonnes publications vraiment spécialisées dans les domaines de l'électronique, électricité, génie civil, informatique etc. Vous pouvez vous abonner pour recevoir leur très bon catalogue. Allez voir plus particulièrement les liens "Hermes Science" et "Tech \& Doc".\\
	\href{http://www.lavoisier.fr}{\color{blue} http://www.lavoisier.fr} (\href{http://www.editions-hermes.fr}{\color{blue} http://www.editions-hermes.fr} + \href{http://www.tec-et-doc.com}{\color{blue} http://www.tec-et-doc.com})
	
	{\Large \ding{170}}{\Large \ding{45}}{\Large \ding{229}} World Scientific Editions - Cet éditeur américainpropose un large éventail de publications scientifiques et de litératures mathématiques de haut niveau (voir "wide range" dans les sections correspondantes sur leur site).\\
	\href{http://www.worldscibooks.com}{\color{blue}http://www.worldscibooks.com}
	
	\section{Associations}

		{\Large \ding{44}} myScience.ch donne un aperçu général de la science, recherche, des universités, entreprises et centres de recherche en Suisse. Des infos pratiques sur les emplois, financement, et la vie en Suisse mais aussi les nouveautés pour chercheurs, scientifiques et académiciens, et tous ceux intéressés par la science. Une grosse proportion d'articles sont en allemand, même dans la version française du site.\\
		\href{http://www.myscience.ch}{\color{blue}http://www.myscience.ch}
		
		{\Large \ding{229}}{\Large \ding{44}}{\Large \ding{73}}\bcdfrance{} L'Union Rationaliste est une association qui vise à promouvoir le role du raisonnement dans les débats intellectuels ou publics, au devant des excès d'irrationel. Elle lute pour permettre à tous d'accéder à une compréhension intelligible du Monde et de la Vie. Elle se bat cont un Etat archaïque et assume son rôle de protection contre l'endoctrinement.\\
		\href{http://www.union-rationaliste.org}{\color{blue}http://www.union-rationaliste.org}
		
		{\Large \ding{45}}{\Large \ding{229}} La société américaine de Mathématiquesest une association de mathématiciens professionels dédiée aux intérêts de la recherche et de l'enseignement des mathématiques. Ses actions prennent la forme de publications gratuites variées, de conférences et de prix remis aux mathématiciens.\\
		\href{http://www.ams.org}{\color{blue}http://www.ams.org}
		
		{\Large \ding{45}}{\Large \ding{229}} La société Americaine de physique fondée le 20 mai 1899, est basée aux USA, est très active dans le domaine de la recherche en physique.\\
		\href{http://www.aps.org}{\color{blue}http://www.aps.org}
		
		{\Large \ding{45}}{\Large \ding{229}}{\Large \ding{170}} La société européene de Physique (EPS) est une organization non lucrative qui promeut la physique en Europe. Elle propose un abonnement au "Europhysicsnews magazine" et bien plus encore.\\
		\href{http://www.eps.org}{\color{blue}http://www.eps.org}

	\pagebreak
	\section{Emplois}
	
	{\Large \ding{73}} Voici une liste de sites web concernant les emplois specifiquement pour les physiciens et mathématiciens suisses... (même si un lien est uniquement pour le territoire US !). D'où les nombreuses descriptions de poste en allemand (...) et le propositions dans le secteur bancaire (qui représentent ~10-15\% de l'emploi en Suisse...). D'autres sont liées aux sites plus génériques, comme le fameux jobup.ch bien connu des suisses.
	
	\begin{itemize}	 
		\item[$-$] \href{http://www.telejob.ch}{\color{blue}http://www.telejob.ch} 
	
		\item[$-$] \href{http://www.jobs.myscience.ch}{\color{blue}http://www.jobs.myscience.ch} 
	
		\item[$-$] \href{http://www.sciencejobs.com}{\color{blue}http://www.sciencejobs.com}
	
		\item[$-$] \href{http://www.math-jobs.ch}{\color{blue}http://www.math-jobs.ch}
	
		\item[$-$] \href{http://www.analyticrecruiting.com}{\color{blue}http://www.analyticrecruiting.com}
	\end{itemize}
	
	\section{Television/Radio}

	{\Large \ding{73}} Un excellent site expliquant, avec des animations 3D,  le fonctionnement d'appareils de la vie courante (moteur, réfrigerateur, pompe, etc.).\\
	\href{http://www.learnengineering.org}{\color{blue}http://www.learnengineering.org}
	
	\bcdfrance{} C'est pas Sorcier: la fameuse émission française de TV qui explique toutes les sciences que les plus jeunes (et moins jeunes) devraient connaître ou découvrir !!\\
	\href{https://www.youtube.com/user/cestpassorcierftv/}{\color{blue}https://www.youtube.com/user/cestpassorcierftv/}
	
	\bcdfrance{} Cité des Sciences : voir des video-conferences, écouter des enregistrement sur des sujets de physique et d'astronomie (et autres) pour le grand public.\\
	\href{http://www.cite-sciences.fr}{\color{blue}http://www.cite-sciences.fr}
	
	\bcdfrance{} Sur Canal Académie, les académiciens, spécialistes en physique, mathématiques, humanitaire, philosophie, sociologie, lois et jurisprudence, économie, politique et finance, histoire, géographie et démographie et sciences politiques partages leurs vues sur l'actualité et les développement de la société.\\
	\href{http://www.canalacademie.com}{\color{blue}http://www.canalacademie.com}

	\pagebreak
	\section{Autres sciences}

	{\Large \ding{52}}{\Large \ding{45}}{\Large \ding{41}}{\Large \ding{44}}\bcdfrance{} site proposant une myriade d'exercices résolus. L'accès est payant avec une durée limitée d'accès... le prix est bas et vaut la peine pour avoir de la qualité.\\
	\href{http://www.web-sciences.com}{\color{blue}http://www.web-sciences.com}
	
	{\Large \ding{52}}{\Large \ding{41}}{\Large \ding{229}}{\Large \ding{44}}{\Large \ding{73}}\bcdfrance{} Futura-Sciences est le site de vulgarisation scientifique, de l'actualité des sciences et technologies, sur une multitude de sujets, des forums scientifiques avec débats et discussions... (le forum de physique est particulièrement bien fourni).\\
	\href{http://www.futura-sciences.com}{\color{blue}http://www.futura-sciences.com}
	
	{\Large \ding{52}}{\Large \ding{41}}{\Large \ding{229}}{\Large \ding{44}}{\Large \ding{73}}\bcdfrance{} Physics Forum est considéré comme la plus grande communauté au monde. Il y a des tonnes de discussions sur divers sujets adressé avec qualité par la communauté. Ce forum est aussi considéré comme partenaire de ce livre (même s'il considèrent ce livre comme spam...) comme je n'ai plus le temps de répondre aux questions par email. \\
	\href{http://www.physicsforums.com}{\color{blue}http://www.physicsforums.com}
	
	{\Large \ding{52}}{\Large \ding{45}}{\Large \ding{41}}{\Large \ding{36}}{\Large \ding{229}}{\Large \ding{44}}{\Large \ding{170}}{\Large \ding{73}}\bcdfrance{} Astrosurf est un portail de liens (annuaire) pour les astronomes amateurs français. Il y a aussi des forums d'astronomie, des annonces, hébergement gratuit de sites d'astronomie, éphémérides et les sites des clubs et associations d'astronomie français.\\
	\href{http://www.astrosurf.com}{\color{blue}http://www.astrosurf.com}
	
	{\Large \ding{52}}{\Large \ding{229}} Le C.E.A. est l'institution française "Commissariat à l'Énergie Atomique". Le site offre les actualités et des documents intéressants dans certains domaines de physique avancée. On peut aussi y trouver des ressources gratuites pour enseignants (diaporamas et affiches).\\
	\href{http://www.cea.fr}{\color{blue}http://www.cea.fr}
	
	{\Large \ding{44}}{\Large \ding{73}} Voici LE site qui, si j'étais un enfant, m'aurait rendu heureux... en ruinant mes parents. C'est plus la partie vente en ligne qui propose des centaines de gadgets et de jeux éducatifs pour les passionnés de science. Attention à ne pas trop dépenser !\\
	\href{http://www.xump.com}{\color{blue}http://www.xump.com}
	
	{\Large \ding{44}}{\Large \ding{73}} Pendant des années, les dessins de S. Harris ont associé humour et sciecne dans un nombre incalculable de magazines, livres, journaux et sites internet scientifiques. Régalez-vous !\\
	\href{http://www.sciencecartoonsplus.com/}{\color{blue}http://www.sciencecartoonsplus.com/}
	
	\pagebreak
	\section{Logiciels/Applications}
	
	{\Large \ding{52}}{\Large \ding{36}}{\Large \ding{170}}{\Large \ding{73}} Le logiciel TeXMaker utilisé pour écrire ce livre en \LaTeX (logiciel tournant sur différents systèmes).\\
	\href{http://www.xm1math.net/texmaker/index.html}{\color{blue}http://www.xm1math.net}
	
	{\Large \ding{52}}{\Large \ding{36}}{\Large \ding{170}}{\Large \ding{73}} Minitab est le logiciel leader pour les ingénieurs travaillant dans l'industrie utilisant des process de contrôle statistique (\SeeChapter{cf la section ingénierie Industrielle}) dans leur travail ou des études statistiques dans le contexte de R\& D.\\
	\href{http://www.minitab.com}{\color{blue}http://www.minitab.com}
	
	{\Large \ding{52}}{\Large \ding{36}}{\Large \ding{170}}{\Large \ding{73}} Isograph est une excellente suite de logiciels pour ingénieurs travaillant dans l'industrie et appliquant des techniques de manitenance préventive (Fiabilité, FMEA) et d'aide à la décision (\SeeChapter{cf la section Théorie des Jeux et Décision}).\\
	\href{http://www.isograph-software.com}{\color{blue}http://www.isograph-software.com}
	
	{\Large \ding{52}}{\Large \ding{36}}{\Large \ding{170}}{\Large \ding{73}} La société ReliaSoft développe, selon moi, le meilleur logiciel du marché concernant la Fiabilité et l'ingénieurie de l'assurance qualité. Leur logiciel est utilisé par les plus grandes entreprises mondiales d'ingénieuries. Parmis ces logiciels, on trouve quelques bijoux (particularlièrement Weibull++).\\
	\href{http://ww.reliasoft.com}{\color{blue}http://ww.reliasoft.com}
	
	{\Large \ding{52}}{\Large \ding{36}}{\Large \ding{73}} JMP est pour moi le meilleur logiciel de design d'expériences. Son pédagogie structurée, sa très bonne documentation (livres) avec preuves mathématiques de l'approche utilisée sont vraiment appréciés.\\
	\href{http://www.jmp.com}{\color{blue}http://www.jmp.com}
	
	{\Large \ding{52}}{\Large \ding{36}}{\Large \ding{170}}{\Large \ding{73}} Site Officiel de l'excellent Maple, logiciel de calcul symbolique utilisé dans de nombreux articles de Sciences.ch. De nombreuses documentations et plugins sont téléchargeables.\\
	\href{http://www.mapleapps.com}{\color{blue}http://www.mapleapps.com}
	
	{\Large \ding{52}}{\Large \ding{36}}{\Large \ding{170}}{\Large \ding{73}}\bcdfrance{} TANAGRA est un puissant logiciel libre de "data mining" (en anglais) pour l'enseignement et la recherche. Il implémente un ensemble de méthodes de data mining (plus de 100) dans l'analyse par exploration statistique de données, machine learning et base de données. C'est mon logiciel préféré de data mining de part sa simplicité d'utilisation et son interface épurée.\\
	\href{http://eric.univ-lyon2.fr/~ricco/tanagra/}{\color{blue}http://eric.univ-lyon2.fr/~ricco/tanagra/}
	
	{\Large \ding{52}}{\Large \ding{36}}{\Large \ding{73}} Site Officiel de MATLAB™, logiciel de calcul et de simulation numérique. Je ne suis pas spécialement fan, mais je dois reconnaître que c'est un outil essentiel dans certains domaines, particulièrement pour la partie SimuLink part et R\&D. C'est un "must have" pour les ingénieurs, tout comme LabView.\\
	\href{http://www.mathworks.com}{\color{blue}http://www.mathworks.com}
	
	\pagebreak
	{\Large \ding{52}}{\Large \ding{36}}{\Large \ding{73}}Site Officiel de la société Statistica. Une référence mondiale dans l'analyse statistique, data mining et le control statistique, qui tient dans un mouchoir de poche avec IBM SPSS.\\
	\href{http://www.statsoft.com}{\color{blue}http://www.statsoft.com}
	
	{\Large \ding{52}}{\Large \ding{36}}{\Large \ding{170}}{\Large \ding{73}}COMSOL Multiphysics (anciennement FEMLAB) est un excellent environnement interactif pour la modélisation industrielle et les applications scientifiques basées sur les équations différentielles partielles (PDEs) utilisant la méthode des éléments finis (plus accessible que ANSYS). Un must pour les ingénieurs et chercheurs (pour ceux qui peuvent payer la licence bien sûr...) !\\
	\href{http://www.comsol.fr}{\color{blue}http://www.comsol.fr}
	
	{\Large \ding{52}}{\Large \ding{36}}{\Large \ding{170}}{\Large \ding{73}} Palisade @Risk est un ensemble de modules de simulation probabiliste pour MS Excel and MS Project (échantillonage Hypercube Monte Carlo et Latin). cela s'intègre dans MS Project et MS Excel, et les modules de calcul basé sur les algorithmes de génétique, réseaux de neuronnes et sa théorie de décision intégrée en font un outil convoité par les dirigeants de grandes entreprises (NASA et Lockheed Martin l'utilisent) et les ingénieurs dans les domaines financier, qualité, gestion de projet et production.\\
	\href{http://www.palisade-europe.com}{\color{blue}http://www.palisade-europe.com}
	
	{\Large \ding{52}}{\Large \ding{36}}{\Large \ding{170}}{\Large \ding{73}} Logiciel d'analyse de décision édité par TreeAge (DATA) permet aux utilisateurs de construire, analyser, d'utiliser des arbres de décision, modèles de Markov et diagrames d'influence. Les modèles DATA incluent des visualisations des aspects quantitatifs et qualitatifs liés au décision du business decisions, afin d'assister les projets pendant l'analyse de risque complexes..\\
	\href{http://www.treeage.com}{\color{blue}http://www.treeage.com}
	
	{\Large \ding{52}}{\Large \ding{36}}{\Large \ding{170}}{\Large \ding{73}} MathType est un puissant éditeur d'équations, (utilisé pour ce site) pour importer/exporter MathML ou TeX ou en application autonome. Avec une barre d'outils, il s'intègre parfaitement à MS Office 2003-2013 suite (Word, Excel, PowerPoint), mais aussi avec 500 autres logiciels. A la différence de \LaTeX 2$\varepsilon$ vous ne passerez pas des heures à résoudre des problèmes de compilation.\\ 
	\href{http://www.dessci.com/en/}{\color{blue}http://www.dessci.com/en/}
	
	{\Large \ding{52}}{\Large \ding{36}}{\Large \ding{73}} Scilab (contraction de Scientific Laboratory) est un logiciel gratuit, développé à l'INRIA Rocquencourt (France). C'est un calculateur numérique qui permet de résoudre numériquement et graphiquement les problèmes rencontrés en mathématiques appliqués, pour les personnes qui ne peuvent s'offrir MATLAB™.\\
	\href{http://www.scilab.org}{\color{blue}http://www.scilab.org}
	
	{\Large \ding{36}}{\Large \ding{73}} Fameux logiciel de géométrie dynamique : Cabri (référence mondiale du domaine) était initialement dédié à l'apprentissage de la géométrie à l'école. Il peut animer des figures géométriques, contrairement à celles dessinées sur un tableau noir. Adapté à la géométrie plane comme à la 3D, c'est l'ancètre te tous les logiciels de géométrie.\\
	\href{http://www.cabri.com}{\color{blue}http://www.cabri.com}
	
	{\Large \ding{36}}{\Large \ding{170}}{\Large \ding{73}}\bcdfrance{} Site d'unenseignant en mathématiques qui a développé un très bon logiciel de calcul pour les graphiques, calculs d'integration, simulations de statistiques et probabilités, facile et ludique en classe (niveau collège).\\
	\href{http://www.patrice-rabiller.fr}{\color{blue}http://www.patrice-rabiller.fr}
	
	{\Large \ding{36}}{\Large \ding{73}} ACDLabs developpe un excellent logiciel "Chemsketch" pour le design et modélisation des molécules (2D, 3D) avec accès à certaines propriétés physiques et chimiques. Très utile pour préparer des articles scientifiques (publications).\\
	\href{http://www.acdlabs.com}{\color{blue}http://www.acdlabs.com}
	
	{\Large \ding{36}}{\Large \ding{73}}{\Large \ding{170}} MolView est une application web Open-Source intuitive pour rendre des molécules encore plus élégantes !\\
	\href{http://molview.org}{\color{blue}http://molview.org}
	
	{\Large \ding{36}}{\Large \ding{73}} Excellent logiciel de pointe pour le design des structures civiles basé sur la méthode des éléments finis.\\
	\href{http://www.scia-online.com}{\color{blue}http://www.scia-online.com}
	
	{\Large \ding{36}}{\Large \ding{73}} Maxima : logiciel libre d'algèbre (à la Maple), sous licence GNU GPLpour le module Macsyma. Ce logiciel de calcul symbolique fut développé à l'origine pour les besoins du département de l'Energie US. Maxima peut faire de l'arithmetique, des polynômes, matrices, de l'integration, dérivation, calcul de séries, limites, résolutions de systèmes d'équations différentielles, etc.\\
	\href{http://maxima.sourceforge.net}{\color{blue}http://maxima.sourceforge.net}
	
	{\Large \ding{36}}{\Large \ding{73}} SPSS (Statistiques Pack pour les Sciences Sociales) est a-priori le logiciel le plus largement utilisé par les entreprises en Suisse, pour l'analyse statistique de données, grace à son impressionnante base de tests et de ses modules business. A cause de son coût et de son propriétaire (IBM), il peut être considéré comme une solution idéale pour les statistiques.\\
	\href{http://www.ibm.com/software/fr/analytics/spss}{\color{blue}http://www.ibm.com/software/fr/analytics/spss}
	
	{\Large \ding{36}}{\Large \ding{170}}{\Large \ding{73}} "R" est un puissant langage et environment de calcul statistque et graphiques. R propose une large variété de statistiques (modélisation linéaire et non-linéaire, tests statistiques classiques, analyse temps-séries, classification, clustering, ...) et des techniques de graphique, le tout hautement extensible. Une des forces de R est la facilité avec laquelle de beaux graphiques de qualité peuvent être produits, incluant symboles mathématiques et formules si nécessaire.\\
	\href{http://www.r-project.org}{\color{blue}http://www.r-project.org}
	
	{\Large \ding{36}}{\Large \ding{73}} LabVIEW (Laboratory Virtual Instrument Engineering Workbench) est un outil puissant de visualisation (langage graphique de programmation) developpé par National Instruments, pour contrôler les instruments de mesures ou robots depuis un PC, et est utilisé par un grand nombre d'entreprise dans le monde (particulièrement les ingénieurs).\\
	\href{http://www.ni.com/labview}{\color{blue}http://www.ni.com/labview}
