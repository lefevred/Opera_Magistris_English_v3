	This directory contains links in 7 categories all relating to science and that I find interesting. I wish to state that under no circumstances I was paid in any form whatsoever for adding links in the list below! You can also found my preferred apps for iPad on my French blog. 
	
	\begin{itemize}	 
		\item[$-$] {\Large \ding{52}} High-Quality website both in terms of design and content
		\item[$-$] {\Large \ding{45}} Content with developments and demonstrations
		\item[$-$] {\Large \ding{41}} Website with forum
		\item[$-$] {\Large \ding{36}} Scientific softwares, sharewares, freewares to download
		\item[$-$] {\Large \ding{229}} Books, publications, magazines, papers (to view or download)
		\item[$-$] {\Large \ding{44}} Cool/funny website
		\item[$-$] {\Large \ding{170}} Favorite website
		\item[$-$] {\Large \ding{73}} Must see 
	\end{itemize}
	
	If there were also three links to highlight of all this is \href{http://www.google.com}{\color{blue} Google}, \href{http://www.wikipedia.com}{\color{blue} Wikipedia} and \href{http://www.youtube.com}{\color{blue} YouTube} that we owe many loans!		

	\pagebreak

	\section{Exact Sciences}

	{\Large \ding{52}}{\Large \ding{45}}{\Large \ding{36}}{\Large \ding{44}}{\Large \ding{170}}{\Large \ding{73}}\bcdfrance{} ChronoMath, small chronology of mathematics, is an educational document that Serge Mehl is constantly renewing since 1988 as a math tutor in French Africa where he was cooperating for many years. More than 450 mathematicians (and their work) are reviewed ...! Must see!\\
	\href{http://www.chronomath.com}{\color{blue} http://www.chronomath.com}
	
	{\Large \ding{41}}{\Large \ding{36}}{\Large \ding{229}}\bcdfrance{} This site offers mathematics news, an encyclopedia with dictionary parts, biographies and formulas, as well as individual files on various mathematical topics and a forum. This site is particularly impressive with respect to the amount of information available for download. \\
	\href{http://www.bibmath.net}{\color{blue} http://www.bibmath.net} 
	
	{\Large \ding{52}}{\Large \ding{45}}\bcdfrance{} A good site dealing with some relevant topics in physics (synophysics). The design could be reviewed in terms of navigation using frames ... but we must still focus on the content quality and it largely predominates.\\
	\href{http://www.sciences.univ-nantes.fr/physique/perso/blanquet/frame3.htm}{\color{blue} http://www.sciences.univ-nantes.fr/physique/perso/blanquet/frame3.htm}
	
	{\Large \ding{52}}{\Large \ding{45}}{\Large \ding{36}}\bcdfrance{} Many PDFs on algebra, geometry and analysis.\\
	\href{http://c.caignaert.free.fr}{\color{blue} http://c.caignaert.free.fr}
	
	{\Large \ding{52}}{\Large \ding{45}}{\Large \ding{36}} arXiv is an archive for electronic preprints of scientific papers. The time elapsing between the time a researcher completes a project and when his work is published in a newspaper may be about a year. At the time scale of research it's a long time. The introduction of the arXiv is therefore a way to overcome this time and cost problem.\\
	\href{http://arxiv.org}{\color{blue} http://arxiv.org}
	
	{\Large \ding{52}}{\Large \ding{45}}{\Large \ding{36}}\bcdfrance{} The multidisciplinary HAL open archive is intended for the deposit and the free dissemination of research level scientifical published and unpublished results and theses from French of foreign educational institutions, public or private laboratories. There is even scientific books published recently by prestigious French publishing houses.\\
	\href{http://hal.archives-ouvertes.fr/}{\color{blue} http://hal.archives-ouvertes.fr/}
	
	\pagebreak
	\section{Publishing/Magazines}

	{\Large \ding{52}}{\Large \ding{41}}{\Large \ding{229}}{\Large \ding{170}}{\Large \ding{73}}\bcdfrance{} We can consider this proposed website as the equivalent of the previous, but for the French engineers. However, it is qualitatively better and there are approximatively the same number of files but more homogeneous. The only regret is perhaps the access to certain elements which is not always easy the first time and also non-free.\\
	\href{http://www.techniques-ingenieur.fr}{\color{blue} http://www.techniques-ingenieur.fr}
	
	{\Large \ding{52}}{\Large \ding{229}}{\Large \ding{170}}{\Large \ding{73}}\bcdfrance{} Absolutely excellent and a must see! Many full courses of the École Polytechnique (France) are published and available for free and to download in PDF format (is it going to last?). About 1,000 educational resources are available in total, whose quality is highly variable, but whose relevance and rarity of subjects is always equal.\\
	\href{http://catalogue.polytechnique.fr}{\color{blue} http://catalogue.polytechnique.fr}
	
	{\Large \ding{52}}{\Large \ding{229}}{\Large \ding{170}}{\Large \ding{73}}\bcdfrance{}  The website itself is not great (which is a pity) but the magazine for which he offers to subscribe (against payment) is well for a visit for people who like mathematics and their actuality.\\
	\href{http://tangente.poleditions.com/}{\color{blue}http://tangente.poleditions.com/}
	
	{\Large \ding{52}}{\Large \ding{229}}{\Large \ding{170}}{\Large \ding{73}}\bcdfrance{} A magazine of the same family as Tangente but at a much higher technical and academic staff by my opinion. The content is particularly oriented on pure mathematics but without direct and explicit applications to physics, engineering or economics.\\
	\href{http://www.quadrature.info}{\color{blue}http://www.quadrature.info}
	
	{\Large \ding{52}}{\Large \ding{229}}{\Large \ding{170}}{\Large \ding{73}} The website itself is not great either (which is a pity, too) but some of the proposed works are just historical!\\
	\href{http://urss.ru}{\color{blue}http://urss.ru}
	
	{\Large \ding{52}}{\Large \ding{229}}{\Large \ding{170}}{\Large \ding{73}}\bcdfrance{} Excellent website with a huge resource (links directory) of electronic documentation dealing with mathematics only in French and English. The site contains links to the pages of the authors of the documents (if not it would require a considerable server ...).\\
	\href{http://mathslinker.chez-alice.fr}{\color{blue}http://mathslinker.chez-alice.fr}
	
	{\Large \ding{229}}\bcdfrance{} This site contains an on-line library of publications in French of the greatest mathematicians of the 20th and 19th century. Must See!\\
	\href{http://matwbn.icm.edu.pl/wyszukiwarka.php}{\color{blue}http://matwbn.icm.edu.pl/wyszukiwarka.php}
	
	{\Large \ding{52}}{\Large \ding{229}}{\Large \ding{170}}{\Large \ding{73}}\bcdfrance{} Reprints of fundamentals works on mathematics, physics, history and philosophy of science.\\
	\href{http://www.gabay.com}{\color{blue}http://www.gabay.com}
	
	\pagebreak
	{\Large \ding{52}}{\Large \ding{229}}{\Large \ding{170}}{\Large \ding{73}}\bcdfrance{} Eyrolles Editions - Excellent website containing numerous works of quality in English and French. This website propose the books of several publishers including Wiley, Springer, etc. It is worthwhile to go take a look in the sections "Mathematics" and "Physics" there are good things... \\
	\href{http://www.eyrolles.com}{\color{blue}http://www.eyrolles.com}
	
	{\Large \ding{52}}{\Large \ding{229}}{\Large \ding{170}}{\Large \ding{73}}\bcdfrance{} Dunod Editions - Propose contemporary literature (under and postgraduate). The books of this publishing house are mostly excellent. Personally they are technically (but not pedagogically) my favorite, because often the developments are very detailed.\\
	\href{http://www.dunod.com}{\color{blue}http://www.dunod.com}
	
	{\Large \ding{229}}{\Large \ding{170}} Springer Editions - Propose scientific literature of Phd level. The site navigation is not easy because the choices are arranged a bit anyhow but if the proposed works are very technical and high level ... very high level.\\
	\href{http://www.springer.de}{\color{blue}http://www.springer.de}
	
	{\Large \ding{229}}{\Large \ding{170}}{\Large \ding{73}}\bcdfrance{} The NUMDAM program, led by MathDoc (UMS 5638 CNRS - UJF) on behalf of CNRS, provides retrospective digitization of mathematics funds published in France.\\
	\href{http://www.numdam.org}{\color{blue}http://www.numdam.org}
	
	{\Large \ding{52}}{\Large \ding{229}}{\Large \ding{170}}{\Large \ding{73}} Wrox Editions - Propose computer development books (expert level). As far as I know (and I think so), this publishing house is the worldwide reference in the field of books about programming languages.\\
	\href{http://www.wrox.com}{\color{blue}http://www.wrox.com}
	
	{\Large \ding{52}}{\Large \ding{170}}{\Large \ding{73}} Cambridge University Press Editions - Propose also postdoc level scientific literature. Somewhat the equivalent of Springer Editions but the website structure is a little bit better. You can particularly appreciate the opportunity to subscribe for free to receive regularly their catalog.\\
	\href{http://www.cambridge.org}{\color{blue}http://www.cambridge.org}
	
	
	{\Large \ding{45}}{\Large \ding{229}}{\Large \ding{170}}{\Large \ding{73}} The Digital Library of France (BNF) offers a free download of the scientific literature (among others) whose copyrights have fallen after 75 years. The download system is not user friendly, but you can found very relevant works. PDFs are often more than 10 MB so take care if you have a slow Internet connexion.\\
	\href{http://gallica.bnf.fr}{\color{blue} http://gallica.bnf.fr}
	
	{\Large \ding{45}}{\Large \ding{229}}{\Large \ding{170}}{\Large \ding{73}} The Physical Review Letters (PRL) search engine (scientific papers) offers against registration and payment (sic) access to scientific research articles in theoretical and experimental physics published for over 100 years. The level of detail of the articles offered is often not very high but this is not the main objective for specialists.\\
	\href{http://prola.aps.org/search}{\color{blue} http://prola.aps.org/search}
	
	\pagebreak
	{\Large \ding{45}}{\Large \ding{229}}{\Large \ding{73}} Lavoisier Editions - Very good publisher offering books in the highly specialized field of electronic, electrical, civil, computer engineering etc. You can also particularly appreciate the opportunity to subscribe for free to receive regularly their great catalogue. See especially the "Hermes Science" and "Tech \& Doc" link of this publisher.\\
	\href{http://www.lavoisier.fr}{\color{blue} http://www.lavoisier.fr} (\href{http://www.editions-hermes.fr}{\color{blue} http://www.editions-hermes.fr} + \href{http://www.tec-et-doc.com}{\color{blue} http://www.tec-et-doc.com})
	
	{\Large \ding{170}}{\Large \ding{45}}{\Large \ding{229}} World Scientific Editions - This american publisher provides provides a wide range of high-level physical sciences and mathematics literature (see the wide range of scientific sections corresponding on their site).\\
	\href{http://www.worldscibooks.com}{\color{blue}http://www.worldscibooks.com}
	
	\section{Associations}

		{\Large \ding{44}} myScience.ch gives an overview of science, research, universities, companies and other research centers in Switzerland. The site provides practical information on jobs, funding and lives in Switzerland and also science news to researchers, scientists, academics and anyone interested in science. It has a higher proportion of articles in German ... even in the French version...\\
		\href{http://www.myscience.ch}{\color{blue}http://www.myscience.ch}
		
		{\Large \ding{229}}{\Large \ding{44}}{\Large \ding{73}}\bcdfrance{} The Rationalist Union aims to promote the role of reason in the intellectual debate as in the public debate, in front of all irrational excesses. It strives to make available to everyone the opportunity to access an intelligible conception of the world and the life. She strike for a State that remains secular and assume its function of protecting against all forms of indoctrination.\\
		\href{http://www.union-rationaliste.org}{\color{blue}http://www.union-rationaliste.org}
		
		{\Large \ding{45}}{\Large \ding{229}} The American Society of Mathematics is an association of professional mathematicians dedicated to the interests of research and teaching of mathematics, what it does in the form of various free publications and conferences, and prizes awarded to mathematicians.\\
		\href{http://www.ams.org}{\color{blue}http://www.ams.org}
		
		{\Large \ding{45}}{\Large \ding{229}} The American Physical Society is a scientific society founded in May 20, 1899, based in the U.S., very active in the field of scientific research in physics.\\
		\href{http://www.aps.org}{\color{blue}http://www.aps.org}
		
		{\Large \ding{45}}{\Large \ding{229}}{\Large \ding{170}} The European Physical Society (EPS) is a non-profit organization whose purpose is to promote physics and physicists in Europe. This society propose a subscription to it's Europhysicsnews magazine and many other things.\\
		\href{http://www.eps.org}{\color{blue}http://www.eps.org}

	\pagebreak
	\section{Jobs}
	
	{\Large \ding{73}} Here is a list of websites for job positions especially for swiss physicists and mathematicians ... (however a link is only for U.S. territory!). So the descriptions are often in German (...) and the job positions for the banking sector (who represents ~10-15\% of employment in Switzerland from what I know ...). Some offers are linked to more general sites like the famous jobup.ch actually well known in Switzerland.
	
	\begin{itemize}	 
		\item[$-$] \href{http://www.telejob.ch}{\color{blue}http://www.telejob.ch} 
	
		\item[$-$] \href{http://www.jobs.myscience.ch}{\color{blue}http://www.jobs.myscience.ch} 
	
		\item[$-$] \href{http://www.sciencejobs.com}{\color{blue}http://www.sciencejobs.com}
	
		\item[$-$] \href{http://www.math-jobs.ch}{\color{blue}http://www.math-jobs.ch}
	
		\item[$-$] \href{http://www.analyticrecruiting.com}{\color{blue}http://www.analyticrecruiting.com}
	\end{itemize}
	
	\section{Television/Radio}

	{\Large \ding{73}} Excellent website explaining with 3D animation many classical equipment of daily life designed and developed by engineering methods (motor, refrigerator , pump, etc.).\\
	\href{http://www.learnengineering.org}{\color{blue}http://www.learnengineering.org}
	
	\bcdfrance{} C'est pas Sorcier: The famous french TV show (modern version in french of "Bill nye the Science guy") about all sciences that almost all young french speaking have know or still discover!!\\
	\href{https://www.youtube.com/user/cestpassorcierftv/}{\color{blue}https://www.youtube.com/user/cestpassorcierftv/}
	
	\bcdfrance{} Cité-Sciences: Viewing video-conferences, or listening to audiotapes on popular topics of physics and astronomy (+ others) popularized.\\
	\href{http://www.cite-sciences.fr}{\color{blue}http://www.cite-sciences.fr}
	
	\bcdfrance{} On Canal Académie, academics, specialists in physics, mathematics, humanities, philosophy, sociology, law and jurisprudence, economics, politics and finance, history, geography and demography and political science will share their thoughts on the news and society developments.\\
	\href{http://www.canalacademie.com}{\color{blue}http://www.canalacademie.com}

	\pagebreak
	\section{Other sciences}

	{\Large \ding{52}}{\Large \ding{45}}{\Large \ding{41}}{\Large \ding{44}}\bcdfrance{} Website with many pages and a huge amount of interesting solved exercises. Unfortunately, this website became paying with access for a limited period ... the price is very low and it may still be appropriate to pay for quality.\\
	\href{http://www.web-sciences.com}{\color{blue}http://www.web-sciences.com}
	
	{\Large \ding{52}}{\Large \ding{41}}{\Large \ding{229}}{\Large \ding{44}}{\Large \ding{73}}\bcdfrance{} Futura-Sciences is website for the popularization of pure and exact sciences with news about sciences and technologies, issues on many topics, scientific forums with debate and discussiosn ... (the physcis forum is especially well attended).\\
	\href{http://www.futura-sciences.com}{\color{blue}http://www.futura-sciences.com}
	
	{\Large \ding{52}}{\Large \ding{41}}{\Large \ding{229}}{\Large \ding{44}}{\Large \ding{73}}\bcdfrance{} Physics Forum is consider as the world's largest physics community. There is a ton of discussions on many subjects answered with quality by the community. This forum is also considered as the partner forum of this e-book (even if they consider my book as spam...) as I don't have time to answer to questions anymore for free by e-mail. \\
	\href{http://www.physicsforums.com}{\color{blue}http://www.physicsforums.com}
	
	{\Large \ding{52}}{\Large \ding{45}}{\Large \ding{41}}{\Large \ding{36}}{\Large \ding{229}}{\Large \ding{44}}{\Large \ding{170}}{\Large \ding{73}}\bcdfrance{} Astrosurf is a portal of links (directory) for French amateur astronomers. There is also astronomy forums astronomy, announcements, free astronomy websites hosting, ephemeris and all the clubs and associations of French astronomy. You can particularly found the famous site of Thierry Lombry (here).\\
	\href{http://www.astrosurf.com}{\color{blue}http://www.astrosurf.com}
	
	{\Large \ding{52}}{\Large \ding{229}} The C.E.A. is the French Commissariat à l'Énergie Atomique. The site offers news and interesting files on particular areas of advanced physics. One can also find free educational resources for teachers (slide shows and posters).\\
	\href{http://www.cea.fr}{\color{blue}http://www.cea.fr}
	
	{\Large \ding{44}}{\Large \ding{73}} Here is a site that, if I was a child, would have made me happy ... and the misery of my parents. It's not so much the content that is interesting but especially the online store (eStore) which offers a hundred gadgets and educational games for passionate of science. Attention to not spend too much!\\
	\href{http://www.xump.com}{\color{blue}http://www.xump.com}
	
	{\Large \ding{44}}{\Large \ding{73}} For years, the cartoons of S. Harris have added humor to innumerable magazines, books, newsletters, ads and web sites in the field of sciences. Enjoy!\\
	\href{http://www.sciencecartoonsplus.com/}{\color{blue}http://www.sciencecartoonsplus.com/}
	
	\pagebreak
	\section{Softwares/Applications}
	
	{\Large \ding{52}}{\Large \ding{36}}{\Large \ding{170}}{\Large \ding{73}} Website of the software TeXMaker used to write this book in \LaTeX (software working on multiple Operating Softwares).\\
	\href{http://www.xm1math.net/texmaker/index.html}{\color{blue}http://www.xm1math.net}
	
	{\Large \ding{52}}{\Large \ding{36}}{\Large \ding{170}}{\Large \ding{73}} Minitab is the leading software for engineers working in industry and applying statistical process control (\SeeChapter{see section Industrial Engineering}) as part of their work or making any statistical study in the context of R\& D.\\
	\href{http://www.minitab.com}{\color{blue}http://www.minitab.com}
	
	{\Large \ding{52}}{\Large \ding{36}}{\Large \ding{170}}{\Large \ding{73}} Isograph is an excellent suite of software for engineers working in industry and applying the techniques of preventive maintenance (reliability, FMEA) and decision support (\SeeChapter{see section Theory Of Games And Decision}).\\
	\href{http://www.isograph-software.com}{\color{blue}http://www.isograph-software.com}
	
	{\Large \ding{52}}{\Large \ding{36}}{\Large \ding{170}}{\Large \ding{73}} The ReliaSoft company develops in my opinion the best softwares on the market for reliability and quality assurance engineers . Their software is used by the best engineering companies worldwied. Overall their software are real small diamonds (particularly Weibull++).\\
	\href{http://ww.reliasoft.com}{\color{blue}http://ww.reliasoft.com}
	
	{\Large \ding{52}}{\Large \ding{36}}{\Large \ding{73}} JMP is for me the best known software for design of experiments at this date. His pedagogical structured and the very good available documentation (books) with mathematical proofs of the approach used by the software is well appreciated.\\
	\href{http://www.jmp.com}{\color{blue}http://www.jmp.com}
	
	{\Large \ding{52}}{\Large \ding{36}}{\Large \ding{170}}{\Large \ding{73}} Official website of the excellent Maple symbolic computation software used for many examples in the different chapters of Sciences.ch. The website also offers lots of documentation and plugins to download.\\
	\href{http://www.mapleapps.com}{\color{blue}http://www.mapleapps.com}
	
	{\Large \ding{52}}{\Large \ding{36}}{\Large \ding{170}}{\Large \ding{73}}\bcdfrance{} TANAGRA is a data mining powerful freeware (in English) for education and research. It implements a set of data mining methods (more than 100) in the field of statistical exploratory data analysis, machine learning and databases. This is my favorite  data mining software because of its simplicity of use and uncluttered user interface.\\
	\href{http://eric.univ-lyon2.fr/~ricco/tanagra/}{\color{blue}http://eric.univ-lyon2.fr/~ricco/tanagra/}
	
	{\Large \ding{52}}{\Large \ding{36}}{\Large \ding{73}} Official website of MATLAB™ calculation and simulation software. I'm not especially a fan but I must admit this is an essential tool in some businesses, especially for the SimuLink part and R\&D Engineering. This is somewhat a "must have" for engineers as well as LabView.\\
	\href{http://www.mathworks.com}{\color{blue}http://www.mathworks.com}
	
	\pagebreak
	{\Large \ding{52}}{\Large \ding{36}}{\Large \ding{73}}Official website of the company Statistica. A worldwide reference in the field of statistical analysis, data mining and statistical process control that stands a priori in a handkerchief with IBM SPSS.\\
	\href{http://www.statsoft.com}{\color{blue}http://www.statsoft.com}
	
	{\Large \ding{52}}{\Large \ding{36}}{\Large \ding{170}}{\Large \ding{73}}COMSOL Multiphysics (formerly FEMLAB) is an excellent interactive environment for the modeling of industrial and scientific applications based on partial differential equations (PDEs) using finite element methods (more eays to use than ANSYS). A must have for engineers and researchers (for those who can afford to buy the license of course ..)!\\
	\href{http://www.comsol.fr}{\color{blue}http://www.comsol.fr}
	
	{\Large \ding{52}}{\Large \ding{36}}{\Large \ding{170}}{\Large \ding{73}} Palisade @Risk is an add-in suite for MS Excel and MS Project for probabilistic simulation (Monte Carlo and Latin Hypercube). It's joint integration with MS Project and MS Excel and its calculations modules based on genetic algorithms, neural networks and its add-in of decision theory makes it a coveted tool for high executive officers of large companies (NASA and Lockheed Martin use it for major projects) and high level engineers in the field finance, quality, project management and production.\\
	\href{http://www.palisade-europe.com}{\color{blue}http://www.palisade-europe.com}
	
	{\Large \ding{52}}{\Large \ding{36}}{\Large \ding{170}}{\Large \ding{73}} The Decision Analysis software edited by TreeAge (DATA) allows users to build, analyse and distribute tree analysis, Markov models and influence diagrams. The DATA models incorporate visually quantitative and qualitative aspects related to business decisions to provide a tool to assist in projects during analysis of complex risks.\\
	\href{http://www.treeage.com}{\color{blue}http://www.treeage.com}
	
	{\Large \ding{52}}{\Large \ding{36}}{\Large \ding{170}}{\Large \ding{73}} MathType is a powerful equation editor (software used for this website) to import/export MathML or TeX or you can run in it stand-alone application. With a toolbar, it also fits perfectly with the programs of the MS Office 2003-2013 suite (Word, Excel, PowerPoint), but also with 500 other softwares (specifically math softwares, or not). At the difference with \LaTeX 2$\varepsilon$ you will not loose hours to solve compatibility and compilation issues.\\ 
	\href{http://www.dessci.com/en/}{\color{blue}http://www.dessci.com/en/}
	
	{\Large \ding{52}}{\Large \ding{36}}{\Large \ding{73}} Scilab (contraction of Scientific Laboratory) is a free software, developed at INRIA Rocquencourt (France). It is a numerical computing environment that enables fast numeric resolutions and graphics commonly encountered in Applied Mathematics for people or small business having not enough money to buy MATLAB™.\\
	\href{http://www.scilab.org}{\color{blue}http://www.scilab.org}
	
	{\Large \ding{36}}{\Large \ding{73}} Official website of the famous dynamic geometry software: Cabri (worldwide reference in this domain) primarily destinated for the learning of geometry in schools. It can animate geometric figures, unlike those drawn on the blackboard. It comes to plane geometry or for geometry in 3D. It is the ancestor of all dynamic geometry software.\\
	\href{http://www.cabri.com}{\color{blue}http://www.cabri.com}
	
	{\Large \ding{36}}{\Large \ding{170}}{\Large \ding{73}}\bcdfrance{} Website of a math teacher who developed a very good freeware for many relevant charts, integration calculations, simulations of statistics and probabilities easily and playfully in the classroom (College level).\\
	\href{http://www.patrice-rabiller.fr}{\color{blue}http://www.patrice-rabiller.fr}
	
	{\Large \ding{36}}{\Large \ding{73}} ACDLabs develops the excellent Chemsketch software for molecules modeling and design (2D, 3D) with access to certain physical and chemical properties. This is a very useful tool also to prepare handouts or scientific articles (publications).\\
	\href{http://www.acdlabs.com}{\color{blue}http://www.acdlabs.com}
	
	{\Large \ding{36}}{\Large \ding{73}}{\Large \ding{170}} MolView is an intuitive, Open-Source web-application to make molecular chemistry science more awesome!\\
	\href{http://molview.org}{\color{blue}http://molview.org}
	
	{\Large \ding{36}}{\Large \ding{73}} Excellent high-level software for the design of civil structures based on finite element methods (not misused in Switzerland for simple or complex projects and studies for engineers).\\
	\href{http://www.scia-online.com}{\color{blue}http://www.scia-online.com}
	
	{\Large \ding{36}}{\Large \ding{73}} Maxima is a computer algebra freeware (a Maple like software), falling under GNU GPL under the package Macsyma, the symbolic calculation software originally developed for the needs of U.S. Department of Energy. Maxima can do arithmetic, polynomials, matrices, integration, derivation, the calculation of series, limits, resolutions of systems of differential equations, etc.\\
	\href{http://maxima.sourceforge.net}{\color{blue}http://maxima.sourceforge.net}
	
	{\Large \ding{36}}{\Large \ding{73}} SPSS (Statistical Package for the Social Sciences) is a priori the most widely used software in companies in Switzerland for statistical analysis of data because it has an impressive number of tests as well as many business packages. Because of its price and its owner (IBM) can be regarded as the premium solution of Statistics.\\
	\href{http://www.ibm.com/software/fr/analytics/spss}{\color{blue}http://www.ibm.com/software/fr/analytics/spss}
	
	{\Large \ding{36}}{\Large \ding{170}}{\Large \ding{73}} R is a free powerful language and environment for statistical computing and graphics. R provides a wide variety of statistical (linear and nonlinear modeling, classical statistical tests, time-series analysis, classification, clustering, ...) and graphical techniques, and is highly extensible. One of R's strengths is the ease with which well-designed publication-quality plots can be produced, including mathematical symbols and formulae where needed.\\
	\href{http://www.r-project.org}{\color{blue}http://www.r-project.org}
	
	{\Large \ding{36}}{\Large \ding{73}} LabVIEW (Laboratory Virtual Instrument Engineering Workbench) is a powerful visual tool (graphical programming language) developed by National Instruments to control measures instrument or robots from a PC and used by a lot of companies worldwide (particularly by engineers).\\
	\href{http://www.ni.com/labview}{\color{blue}http://www.ni.com/labview}