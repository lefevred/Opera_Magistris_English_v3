	\pichskip{15pt}% Horizontal gap between picture and text
	\parpic[l][t]{
	  \begin{minipage}{55mm}
    	\fbox{\includegraphics[scale=0.72]{img/medaillons/isoz.eps}}
	  \end{minipage}
	}
	\textbf{Vincent ISOZ}\\
	Année de naissance : 1978\\
	Nationalités : Française, Suisse\\
	Habite à Lausanne (Suisse)
	
	Ingénieur HES-SO en Physique Appliquée (2001)\\
	Equivalence internationale : Baccalauréat universitaire (L3) en Sciences

	\faWikipediaW ikipedia Contributions \& Donations
	
	\href{https://fr.scribd.com/collections/3709887/My-e-books}{{\color{blue} my e-Books}}, \href{https://www.video2brain.com/fr/formateur/vincent-isoz/}{{\color{blue} mes Cours}}
	\\
	\href{https://www.linkedin.com/in/vincentisoz}{\faLinkedinSquare{}lln/vincentisoz}\\
	\href{https://www.instagram.com/vincent.isoz/}{\faInstagram{}in/vincent.isoz}
	\\
	\href{https://www.facebook.com/groups/1793543747588689/}{\faFacebook{}fb/opera.magistris}\\
	\href{https://github.com/vincentisoz/opera_magistris/}{\faGithub{}git/opera.magistris}\\\\
	
	\textbf{Currently}
	
	Employé chez \href{http://www.scientific-evolution.com}{{\color{blue} Scientific Evolution Sàrl}}, specializé dans le consulting et la formation de hauts-potentiels, plus spécifiquement dans l'analyse quantitative en gestion de projet, qualité, ingénieurie, finance et choix décisionel (mon activité principale est de faire connaître les modèles mathématiques avancés aux sociétés de consulting et aux entreprises industrielles). Incidentally, heureux formateur et manager en R\&D, j'amasse depuis tout jeune d'informations et d'articles, je suis désormais passionné par tout ce qui touche aux Mathématiques Appliquées dans la gestion d'organisations (Commerce, Etats), TQM (Total Quality Management) avec Post-Tayloriste et logiciels associés.
	
	Consultant en Mathématiques, j'encadre des analystes quantitatifs (doctorants et postdoctorants) et de jeunes étudiants haut-potentiels en français, anglais et allement. Je suis aussi auteur de quelques ouvrages en français et anglais, dans les domaines suivants :
	\pagebreak
	\begin{itemize}	 
		\item[$\bullet$] statistical process control (parametric and non-parametric methods)
		\item[$\bullet$] predictive modeling / advanced decision (decision trees, Markov chains)
		\item[$\bullet$] operations Research (quasi-Newton, simplex, genetic algorithms, algorithm GRG)
		\item[$\bullet$] data mining (neural networks, PCA, CA, regressions, scoring, clustering, etc.)
		\item[$\bullet$] risk modeling in project management and corporate finance (monte carlo, bootstrapping)
		\item[$\bullet$] project management (best practices, theoretical models EFQM+Six Sigma, MS Office Project)
		\item[$\bullet$] ISO 9001:2008, 5807:1985, 10015:1999, 31000+31010:2009, 8258:1991, 10017:2003
		\item[$\bullet$] Adobe Photoshop and Illustrator
		\item[$\bullet$] 12 applications of the Microsoft Office System (Project, Visio, SharePoint, Access, etc.)
	\end{itemize}
	A ce jour, je suis intervenu dans quelques 200 entreprises, notament $\sim$10 des Fortune 500 (d'après la liste de 2009) ainsi que trois universitiés et écoles d'ingénieur en Suisse, pour des formations et simulations de risque stochastique appliqués aux bases de données. Formation en cours particuliers de reponsables de multinationnales.

	Le consulting me convient bien, car cela m'offre beacoup d'opportunités de voir comment différentes entreprises fonctionnent, leur manière de traiter les problèmes, d'apprendre aussi beaucoup, et d'être impliqué dans de nombreux projets et défis intéressants.

	Allez visioner mon interview chez Video2Brain \href{http://www.youtube.com/watch?v=nOYwENyVPJQ}{{\color{blue}here}} et une longue video de formation sur les logiciels scientifiques chez \href{http://www.alphorm.com}{{\color{blue}alphorm.com}}.

\textbf{Particularities}	

	Je suis partisan de la mise en place de process, de l'utilisation des standards internationaux et des chartes qualité dans le monde professionel. Je mets un point d'honneur à m'assurer du bien-être et du développement des employés, lorsqu'ils ont un travail correspondant à leur savoir-faire. Je n'hésite pas à dire objectivement les problèmes d'organisation que peuvent avoir mes clients.
	
	D'autre part, je suis passionné d'informatique, de mathématiques appliquées (dans tous les domaines possibles), les systèmes et le partage de connaissance. J'ai toujours des tonnes d'idées en informatique et en mathématiques appliquées, mais jamais assez de temps pour les concrétiser ! J'ai horreur de réinventer la roue, ou de demander à mes collègues de le faire.

\textbf{Studies}	

	Je suis diplômé de l'HES-SO\footnote{Université des Sciences appliquées et Arts - Suisse ouest} dans le domaine de la physique appliquée, option ingénierie nucléaire. J'ai obtenu mon diplome à l'École Polytechnique Fédérale de Lausanne (EPFL-Suisse), dans le departement de physique expérimentale, sur la matère condensée dans un potentiel magnétique.
	
	 J'a également fait un petit projet sur les sondes Langmuir, à l'école d'ingénieurs de Genève, pour les plamas faiblement et fortement ionisés.
	
	Mes connaissances générales d'ingénieur HES en Physique Appliquée me permet de collaborer avec de différents profils professionels (physiciens, chimistes, docteurs, electriciens, mecaniciens, informaticiens, financiers, etc.) afin d'apporter un assistance théorique et technique necessaire à l'accomplissement de projets expérimentaux.

	Le cursus en phisique appliquée intègre des connaissances scientifiques en mathématiqus, physique générale, physique et chimie nucléaire, radiations, acoustique, électricité, électroniques, micro-technologie, analyse numerique et informatique, ainsi que la physique des matériaux.

\textbf{Professional skills (hard \& quant skills)}

Actuellement, interventions régulières, en tant que consultant, pour de nombreuses entreprises nationales et internationales et  administrations dans les domaines suivants :
\begin{itemize}	 
	\item[$\bullet$] Mathématiques appliquées : Statistiques générales, Contrôle statistique et optimisation de process et de procédures, modélisation financiaire, ingénieurie Reliability, Aide à la Décision, Analyse de données

	\item[$\bullet$] Gestion de projet, gestion de Qualité, gestion d'approvisionement, Gestion de documents, Gestion de formation, gestion de risque

	\item[$\bullet$] Developpement (langages utilisés dans divers projets) : XML, XSL, XSD, ADA, C++, C\#, PHP, ASP 3.0, (X)HTML/DHTML, Javascript, VBA, VBScript, SQL (on Oracle/MySQL/Access/SQL Server), RibbonX, XPath, CSS

	\item[$\bullet$] Multimedia : Macromedia Flash, Macromedia Dreamweaver, Adobe Photoshop, Adobe Illustrator, Adobe Acrobat, Oracle UPK

	\item[$\bullet$] MS Office : Word, Excel (+PowerPivot/Power Map/Power View/DAX), Outlook, Visio, PowerPoint, Publisher, FrontPage, InfoPath, Project, Project Server, SharePoint Server Enterprise, Communicator (Lync), Mind Manager, Camtasia

	\item[$\bullet$] Mathematiques : MathType, MATLAB™, Maple, SPSS, SAS EG, R, Minitab, Tanagra, LaTeX and LabView (même si je ne l'ai pas utilisé depuis longtemps)
\end{itemize}
\textbf{Langues}

Français : langue maternelle
Anglais : oral et écrit (utilisé quotidiennement)
German : oral et écrit (utilisé quotidiennement)
Italien : en cours d'apprentissage...

\textbf{Connaissances informatiques}

De multiples connaissances avancées acquises en auto-didacte en développement sur Microsoft, Apple, Unix et les plateformes Linux. Je réalise de formations sur divers systèmes et logiciels (même si certains ne sont plus utilisés depuis quelques temps...):

MS Windows 3.x/9x/2000/XP (Home, Pro), IIS, Apache, SQL Server, Oracle, MS-DOS (QBasic), MS Word, MS Excel (PowerPivot, Power Map, PowerView, DAX), MS Publisher, MS FrontPage, MS InfoPath, MS OneNote, MS PowerPoint, MS Project (Server), MS Visio, MS Access, MS Outlook, SharePoint, Lotus Notes, AutoCAD, Maple, MATLAB™, LabView, MM Dreamweaver, Swish, Swift, MM Flash, Adb Photoshop, Adb Illustrator, Adb Acrobat, 3D Studio Max, Installer Vise, Crystal Reports, MiniTab, SAS, R, @Risk Palissade, Mindjet Mind Manager,...

Assembler Motorola 680xx, C++, VBA, VB.Net, C\#, ASP 3, ASP.Net, ADO .Net, Ada, QBASIC, BASIC, Pascal, HTML, XHTML, DHTML, CSS, PHP, VBScript, JavaScript, XML, XSL, XPath, XSD, RibbonX, SQL, Pascal, MySQL, ActionScript, AJAX, SMIL,...

Quelques concepts en réseaux informatiques (administrateur 1er niveau).

\textbf{Loisirs}

Etudes et developpements en mathématiques appliquées, philosophie, informatique (web, 3D, developpement, graphisme), économie, choix décisionel, modélisation du risque, guitare électrique. Apprendre, apprendre, apprendre et encore... apprendre !

\textbf{Films favoris}

Home (Yann Arthus-Bertrand), Will Hunting (Gus Van Sant), Kill Bill (Quentin Tarantino), A beautiful mind (Ron Howard), Gladiator (Ridley Scott), Persepolis (Marjane Satrapi, Vincent Paronnaud), Castle in the sky (Hauru no Ugoku Shiro), Jason Bourne (Doug Liman, Paul Greengrass), Batman (Tim Burton), Amadeus (Milos Forman), Cyrano (Jean-Paul Rappenau), 8 Miles (Eminem), Ruby Sparks (Jonathan Dayton, Valerie Faris), etc.

\textbf{Compositeurs/Interprètes favoris}

Queen, Prince, Micheal Jackson, Wolfgang Amadeus Mozart, Gioachino Rossini, Daft Punk, Stephan Heicher, Francis Cabrel

\textbf{Auteurs favoris}

Bernard Weber, Trinh Xuan Thuan, Hubert Reeves, Edmond Rostand, Alejandro Jodorowsky, Akira Toriyama, Masashi Kishimoto

\textbf{+Plus...}

	Suite à mon diplôme en 2001, j'ai d'abord pensé à continuer par l'université, devenir enseignant, mais après réflexion, donner le même cours durant 40 ans ne m'inspirait guère (je m'ennuie assez vite...). J'ai finalement choisi de devenir formateur en entreprise, afin de pouvoir travailler dans plusieurs domaines (administration, ingénierie, management, modélisation, design, developpement, assemblage, etc..). J'aime que la technologie évolue très rapidement et que mes auditeurs (employés et dirigeants d'entreprise parlant Allemand, Anglais ou Français) et parfois des études de cas très complexes, sur d'importantes stratégies financières à fort impact (quelques millions, voire millards de dollars).

	Passioné par tout ce qui touche aux mathématiques appliquées depuis mes 12 ans, je souhaitais retrouver cette passion dans mon environnement professionnel. Pour avoir l'agréable et l'utile (l'inverse est également vrai) et après avoir constaté le manque de matériel éducatif de qualité sur internet, en français (au début des années 2000) à propos de la physique théorique, j'ai décidé de transposer ce que j'apprenais chaque jour dans cet ouvrage. Je souhaite pouvoir continuer aussi loin que possible, j'ai toujours énormément à apprendre et à partager, et le savoir est si vaste...

A coté de ce livre, j'ai réalisé bon nombre de formations électroniques, sur divers produits et méthodes. Tout ceci peut être trouvé sur les sites internet spécialisés (simplement en cherchant mon nom).
\begin{flushright}
\includegraphics[width=192pt,height=60pt]{img/paraphe.eps}
\end{flushright}
