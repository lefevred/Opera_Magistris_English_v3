	%to make section start on odd page
	\newpage
	\thispagestyle{empty}
	\mbox{}
	\section{Impressum}	
	\subsection{Use of content}

	The contents of this book are elaborated by a development process by which volunteers reach a consensus. This process that brings together volunteers, research also the point of view of people interested in the topics of this book. The person in charge of this book administers the process and establishes rules to promote fairness in the consensus approach. It is also responsible for drafting the text, sometime for testing/evaluating or independently verifying the accuracy or completeness of the presented information.

	We decline no responsibility for any injury, damage or any other kind, special, incidental, consequential or compensatory, arising from the publication, application or reliance on the content of this book. We make no express or implied warranty on the accuracy or completeness of any information published in this book, and do not guarantee that the information contained in this book meet any specific need or goal of the reader. We do not guarantee the performance of products or services of one manufacturer or vendor solely by virtue of this book content.
	
	The technical descriptions, procedures, and computer programs in this book have been developed without care, therefore they are provide without warranty of any kind. We make also no warranties that the equations, programs, and procedures in this books or its associated software are free of error, or are consistent with any particular standard of merchantability, or will meet your requirements for any particular application. They should not be relied upon for solving a problem whose incorrect solution could result in injury to a person or loss of property. Any use of the content of this book as at the reader's own risk. The authors, redactors, and publisher disclaim all liability for direct, incidental, or consequent damages resulting form use of the content of this book or the associated software.

	By publishing texts, it is not the intention of this book to provide services on behalf of any person or entity or performing any task to be accomplished by any person or entity for the benefit of a third party. Anyone using this book should rely on its own independent judgment or, where that is appropriate, seek the advice of a qualified expert to determine how to exercise reasonable care under all circumstances. The information and standards on the topics covered by this book may be available from other sources that the reader may wish to visit in search of points of view or additional information not covered by the contents of this book.

	We have no power in order to enforce compliance with the contents of this book, and we do not undertake to monitor or enforce such compliance. We have no certification, testing or inspection activity of products, designs or installations for safety or health of persons and property. Any certification or other statement of compliance regarding information relating to health or safety of persons and property, mentioned in this book, cannot possibly be attributed to the content of this book and remains under the responsibility of the certification center or the concerned reporter.

	\pagebreak	
	\subsection{How to use this book}

	At the university level, this book can be used for a Ph.D., graduate level or advanced undergraduate level seminar in many exact and pure sciences fields. The seminars where we use this material is part of \textbf{Scientific Evolution Sàrl} program, where the trainees typically already have taken undergraduate or graduate courses in their respective specialization. In reality this books also aims to cover the full Kindergarten to PhD curriculum.

	Because the methods of Applied Mathematics are learned by practice and experience, we view a seminar on Applied Mathematics as a learning-by-doing (project oriented) seminar. We structure our mathematical modelling seminars around a set of problems that require the trainee to construct models that help with planning and decision making. The imperative is that the models should be consistent with the theory and back-tested. To fulfill this imperative, it is necessary for the trainee to combine mathematical theory with modeling. The result is that the trainee learns the theory, and more importantly, learns how that theory is applied and combined in the real world. The ability to criticize and identify limitations of dangerous mathematical tools is the most valuable feature of our seminars.

	The problems with solutions in this book provide the opportunity to apply the text material to a comprehensive set of fairly realistic situations. By the end of the seminars the trainees will have enhanced their skills and knowledge of the most important theoretical and computing tools. These are valuable skills that are in demand by the businesses at the highest levels.

	It is very difficult to cover all the material in this book in a semester. It takes a lot of time to explain the concepts to the trainees. The reader is encouraged to pick and choose which topics will be covered during the term. It is not necessary strictly necessary to cover them in sequence but it can help in a significant way?

	In a nutshell, this book offers you a wide variety of topics that are amenable to modeling. All are practical.
	
	\subsubsection{Ancilliaries}
	We offer an array of ancilliaries for students, instructors and practitioners.
	
	First there are some free companion eBooks and tools in French and English written by Vincent ISOZ \& Daname KOLANI for the people that want to put in practice the theory presented in this book.
	
	Here is the list:
	\begin{itemize}
		\item MATLAB™ in English (1,339 pages):\\ \href{http://www.sciences.ch/htmlfr/php/cliccount/click.php?id=319}{http://www.sciences.ch/dwnldbl/divers/Matlab.pdf}
		\item Maple in French (99 pages):\\ \href{http://www.sciences.ch/dwnldbl/divers/Maple.pdf}{http://www.sciences.ch/dwnldbl/divers/Maple.pdf}
		\item \textsf{R} in French (1,626 pages):\\ \href{http://www.sciences.ch/htmlfr/php/cliccount/click.php?id=313}{http://www.sciences.ch/dwnldbl/divers/R.pdf}
		\item Minitab in French (1,092 pages):\\ \href{http://www.sciences.ch/htmlfr/php/cliccount/click.php?id=282}{http://www.sciences.ch/dwnldbl/divers/Minitab.pdf}
		\item Scientific Linux installation \& Configuration  (211 pages):\\ \href{http://www.sciences.ch/dwnldbl/divers/ScientificLinux.pdf}{http://www.sciences.ch/dwnldbl/divers/ScientificLinux.pdf}
	\end{itemize}
	\begin{center}
		\includegraphics[scale=0.75]{img/books/matlab.jpg}
		\includegraphics[scale=0.75]{img/books/maple.jpg}
		\includegraphics[scale=0.75]{img/books/r.jpg}
		\includegraphics[scale=0.75]{img/books/minitab.jpg}
		\includegraphics[scale=0.75]{img/books/scientificlinux.jpg} 
	\end{center}	
		
	 In second we offer a few Quizzes and Flashcards in French and English to challenge your students or just yourself with the rest of the world:
	 \begin{itemize}
		\item MATLAB™ Basics L1 Challenge level in French (100 questions)\\ \url{http://www.scientific-evolution.com/qcm/start_session/a73647cf3b/}
		
		\item Astronomy/Astrophysics H1 Challenge level in English (100 questions):\\ \url{http://www.scientific-evolution.com/qcm/start_session/ffd0810fa0/}
		
		\item Greek Letter Flashcards (48 cards):\\
		\url{http://www.scientific-evolution.com/qcm/fr/start_session/6d9f1fef90/}
		
		\item Common Derivatives Flashcards (29 cards):\\
		\url{http://www.scientific-evolution.com/qcm/fr/start_session/c15a40f2c4/}
		
		\item Common Primitives Flashcards (60 cards):\\
		\url{http://www.scientific-evolution.com/qcm/fr/start_session/ccfc20fdef/}
		
		\item Common Trigonometric Identities Flashcards (68 cards):\\
		\url{http://www.scientific-evolution.com/qcm/fr/start_session/882f9696cd/}
		
		\item \LaTeX{} L3 Challenge level in French (100 questions):\\ \url{http://www.scientific-evolution.com/qcm/fr/start_session/ff1e1d1b91/}
		
		\item R Software 3.1.2 L3 Challenge level in French (100 questions):\\ \url{http://www.scientific-evolution.com/qcm/fr/start_session/2a6fca7473/}
		
		\item C++ L3 Challenge level in French (100 questions):\\
		\url{http://www.scientific-evolution.com/qcm/fr/start_session/e031ce4b43/}
	\end{itemize}
	
	And as any technical book should have a forum, the reader ca go through this link for any discussions about the content of this book:
	\begin{center}
		\url{https://www.physicsforums.com}
	\end{center}
	For those who prefer social networks we have also a dedicated  Facebook group:
	\begin{center}
		\faFacebook{} \href{https://www.facebook.com/groups/1793543747588689/}{https://www.facebook.com/groups/opera.magistris}
	\end{center}
	Or for more fun (science pics, quotes, jokes, videos, etc.) there is also an associated Instagram account:
	\begin{center}
		\faInstagram{} \href{https://www.instagram.com/opera.magistris/}{https://www.instagram.com/opera.magistris/}
	\end{center}
	And a collection of a selection of what we consider a interesting scientific videos on our YouTube channel:
	\begin{center}
		\faYoutube{} \href{https://www.youtube.com/user/AdminSciences}{https://www.youtube.com/user/AdminSciences}
	\end{center}

	As for this book, the companion books above are only samples of the complete one. The full version with \underline{perpetual free updates} are available for the price of \$ 299.- each and for \$ 499.- you get the exercise files and \LaTeX{} sources (for information on purchase you can simply send me an {\href{mailto:isoz@sciences.ch}{{\color{blue}email}}}).

	Because this book mainly focus on mathematical aspect of physical phenomena we can only strongly recommend to the reader an another free book that is in our point of view actually the best one that focus on the popular science aspect of the subjects that we will cover:
	\begin{center}
	Motion Mountain by Dr. Christoph Schiller: \url{http://www.motionmountain.net}
	\end{center}
	\begin{center}
		\includegraphics[scale=0.9]{img/books/motion_mountain.jpg}
	\end{center}

	\pagebreak
	\subsection{Data Protection}
	When looking at information on the Internet companion site (Sciences.ch), some data are automatically saved. We try to save as less as possible data and as brief as possible. Wherever we can, we ave only anonymous data. We undertake to process the data you send us personally with the utmost diligence.

	However, your IP address and the source page that takes you on Sciences.ch and the associated keywords, are freely available to everybody here for the current month. After which detailed data are destroyed. You can object at any time in the publication of your data by contacting us.

	\subsection{Use of data}

	Your data are only used for sending the Sciences.ch newsletter. Communication of personal data (except the e-mail address, title and name) is optional. When registering for the newsletter, you can of course specify an alternate address and/or a fictitious name.

	\subsection{Data transmission}

	We will never sell or commercialize the data of our customers or interested parties and will never affects the rights of the person. In addition, we will not rent mailing lists and will not send you advertising from third parties or on our behalf.

	\subsection{Agreement}

	When you provide us personal information, you authorize us to save them and use them within the meaning of the Swiss Federal Law on Data Protection. If you ask us not to send you emails, we are obliged, in your interest, save your e-mail in an internal negative list.
	
	\subsection{Errata}
	Altought we have taken every care to ensure the accuracy of our content, mistakes do happen. If you find a mistake in this boo - maybe a mistake in the text, scripts or illustrations - we would be grateful if you would report this to us. By doing so, you can save other readers from frustration and help us improve subsequent versions of this book. Our e-mail is given on the footer every page of this book. Once your errata are verified, your submissions will be accepted and the error will be visible on the change log of update versions.

	%to make section start on odd page
	\newpage
	\thispagestyle{empty}
	\mbox{}
	\section{License}
	The entire contents of this book is subject to the GNU Free Documentation License, which means:
	\begin{itemize}
			\item[$\bullet$] that everyone has the right to freely use the texts for non-commercial usage (Google Ads or any equivalent being considered as a commercial usage!)
			\item[$\bullet$] that any person is authorized to broadcast items for non-commercial usage (Google Ads or any equivalent being considered as a commercial usage!)
			\item[$\bullet$] that anyone can freely edit the texts for non-commmercial usage (Google Ads or any equivalent being considered as a commercial usage!)
	\end{itemize}
	
	and bla bla bla...

	in accordance with the license described below: 

	\begin{center}
	Version 1.1, March 2000
		
	Copyright (C) 2000 Free Software Foundation, Inc. 59 Temple Place, Suite 330, Boston, MA 02111-1307 USA Everyone is permitted to copy and distribute verbatim copies of this license document, but changing it is not allowed. 
	\end{center}

	\subsection{Preamble} 

	The purpose of this License is to make a manual, textbook, or other written document "free" in the sense of freedom: to assure everyone the effective freedom to copy and redistribute it, with or without modifying it only a non-commercial purpose. Secondarily, this License preserves for the author and publisher a way to get credit for their work, while not being considered responsible for modifications made by others.

	This License is a kind of "copyleft", which means that derivative works of the document must themselves be free in the same sense. It complements the GNU General Public License, which is a copyleft license designed for free software. 

	We have designed this License in order to use it for manuals for free software, because free software needs free documentation: a free program should come with manuals providing the same freedoms that the software does. But this License is not limited to software manuals; it can be used for any textual work, regardless of subject matter or whether it is published as a printed book. We recommend this License principally for works whose purpose is instruction or reference. 

	\subsection{Applicability and Definitions}
	This License applies to any manual or other work that contains a notice placed by the copyright holder saying it can be distributed under the terms of this License. The "Document", below, refers to any such manual or work. Any member of the public is a licensee, and is addressed as "you". 

	A "Modified Version" of the Document means any work containing the Document or a portion of it, either copied verbatim, or with modifications and/or translated into another language. 

	A "Secondary Section" is a named appendix or a front-matter section of the Document that deals exclusively with the relationship of the publishers or authors of the Document to the Document's overall subject (or to related matters) and contains nothing that could fall directly within that overall subject. (For example, if the Document is in part a textbook of mathematics, a Secondary Section may not explain any mathematics.) The relationship could be a matter of historical connection with the subject or with related matters, or of legal, commercial, philosophical, ethical or political position regarding them. 

	The "Invariant Sections" are certain Secondary Sections whose titles are designated, as being those of Invariant Sections, in the notice that says that the Document is released under this License. 

	The "Cover Texts" are certain short passages of text that are listed, as Front-Cover Texts or Back-Cover Texts, in the notice that says that the Document is released under this License. 

	A "Transparent" copy of the Document means a machine-readable copy, represented in a format whose specification is available to the general public, whose contents can be viewed and edited directly and straightforwardly with generic text editors or (for images composed of pixels) generic paint programs or (for drawings) some widely available drawing editor, and that is suitable for input to text formatters or for automatic translation to a variety of formats suitable for input to text formatters. A copy made in an otherwise Transparent file format whose markup has been designed to thwart or discourage subsequent modification by readers is not Transparent. A copy that is not "Transparent" is named "Opaque". 

	Examples of suitable formats for Transparent copies include plain ASCII without markup, Texinfo input format, LaTeX input format, SGML or XML using a publicly available DTD, and standard-conforming simple HTML designed for human modification. Opaque formats include PostScript, PDF, proprietary formats that can be read and edited only by proprietary word processors, SGML or XML for which the DTD and/or processing tools are not generally available, and the machine-generated HTML produced by some word processors for output purposes only. 

	The "Title Page" means, for a printed book, the title page itself, plus such following pages as are needed to hold, legibly, the material this License requires to appear in the title page. For works in formats which do not have any title page as such, "Title Page" means the text near the most prominent appearance of the work's title, preceding the beginning of the body of the text. 

	\subsection{Verbatim Copying} 
	You may copy and distribute the Document in any medium, noncommercially, provided that this License, the copyright notices, and the license notice saying this License applies to the Document are reproduced in all copies, and that you add no other conditions whatsoever to those of this License. You may not use technical measures to obstruct or control the reading or further copying of the copies you make or distribute. However, you may accept compensation in exchange for copies. If you distribute a large enough number of copies you must also follow the conditions in section 3. 

	You may also lend copies, under the same conditions stated above, and you may publicly display copies. 

	\subsection{Copying in Quantity}

	If you publish printed copies of the Document numbering more than 100, and the Document's license notice requires Cover Texts, you must enclose the copies in covers that carry, clearly and legibly, all these Cover Texts: Front-Cover Texts on the front cover, and Back-Cover Texts on the back cover. Both covers must also clearly and legibly identify you as the publisher of these copies. The front cover must present the full title with all words of the title equally prominent and visible. You may add other material on the covers in addition. Copying with changes limited to the covers, as long as they preserve the title of the Document and satisfy these conditions, can be treated as verbatim copying in other respects. 

	If the required texts for either cover are too voluminous to fit legibly, you should put the first ones listed (as many as fit reasonably) on the actual cover, and continue the rest onto adjacent pages. 

	If you publish or distribute Opaque copies of the Document numbering more than 100, you must either include a machine-readable Transparent copy along with each Opaque copy, or state in or with each Opaque copy a publicly-accessible computer-network location containing a complete Transparent copy of the Document, free of added material, which the general network-using public has access to download anonymously at no charge using public-standard network protocols. If you use the latter option, you must take reasonably prudent steps, when you begin distribution of Opaque copies in quantity, to ensure that this Transparent copy will remain thus accessible at the stated location until at least one year after the last time you distribute an Opaque copy (directly or through your agents or retailers) of that edition to the public. 

	It is requested, but not required, that you contact the authors of the Document well before redistributing any large number of copies, to give them a chance to provide you with an updated version of the Document. 

	\subsection{Modifications}

	You may copy and distribute a Modified Version of the Document under the conditions of sections 2 and 3 above, provided that you release the Modified Version under precisely this License, with the Modified Version filling the role of the Document, thus licensing distribution and modification of the Modified Version to whoever possesses a copy of it. In addition, you must do these things in the Modified Version: 
	
	\begin{itemize}
		\item Use in the Title Page (and on the covers, if any) a title distinct from that of the Document, and from those of previous versions (which should, if there were any, be listed in the History section of the Document). You may use the same title as a previous version if the original publisher of that version gives permission. 

		\item List on the Title Page, as authors, one or more persons or entities responsible for authorship of the modifications in the Modified Version, together with at least five of the principal authors of the Document (all of its principal authors, if it has less than five). 

		\item State on the Title page the name of the publisher of the Modified Version, as the publisher. 

		\item Preserve all the copyright notices of the Document. 

		\item Add an appropriate copyright notice for your modifications adjacent to the other copyright notices. 

		\item Include, immediately after the copyright notices, a license notice giving the public permission to use the Modified Version under the terms of this License, in the form shown in the Addendum below. 

		\item Preserve in that license notice the full lists of Invariant Sections and required Cover Texts given in the Document's license notice. 

		\item Include an unaltered copy of this License. 

		\item Preserve the section entitled "History", and its title, and add to it an item stating at least the title, year, new authors, and publisher of the Modified Version as given on the Title Page. If there is no section entitled "History" in the Document, create one stating the title, year, authors, and publisher of the Document as given on its Title Page, then add an item describing the Modified Version as stated in the previous sentence. 

		\item Preserve the network location, if any, given in the Document for public access to a Transparent copy of the Document, and likewise the network locations given in the Document for previous versions it was based on. These may be placed in the "History" section. You may omit a network location for a work that was published at least four years before the Document itself, or if the original publisher of the version it refers to gives permission. 

		\item In any section entitled "Acknowledgements" or "Dedications", preserve the section's title, and preserve in the section all the substance and tone of each of the contributor acknowledgements and/or dedications given therein. 

		\item Preserve all the Invariant Sections of the Document, unaltered in their text and in their titles. Section numbers or the equivalent are not considered part of the section titles. 

		\item Delete any section entitled "Endorsements". Such a section may not be included in the Modified Version. 

		\item Do not retitle any existing section as "Endorsements" or to conflict in title with any Invariant Section.

		\item If the Modified Version includes new front-matter sections or appendices that qualify as Secondary Sections and contain no material copied from the Document, you may at your option designate some or all of these sections as invariant. To do this, add their titles to the list of Invariant Sections in the Modified Version's license notice. These titles must be distinct from any other section titles. 

		\item You may add a section entitled "Endorsements", provided it contains nothing but endorsements of your Modified Version by various parties--for example, statements of peer review or that the text has been approved by an organization as the authoritative definition of a standard. 

		\item You may add a passage of up to five words as a Front-Cover Text, and a passage of up to 25 words as a Back-Cover Text, to the end of the list of Cover Texts in the Modified Version. Only one passage of Front-Cover Text and one of Back-Cover Text may be added by (or through arrangements made by) any one entity. If the Document already includes a cover text for the same cover, previously added by you or by arrangement made by the same entity you are acting on behalf of, you may not add another; but you may replace the old one, on explicit permission from the previous publisher that added the old one. 

		\item The author(s) and publisher(s) of the Document do not by this License give permission to use their names for publicity for or to assert or imply endorsement of any Modified Version.
	\end{itemize} 

	\subsection{Combining Documents}

	You may combine the Document with other documents released under this License, under the terms defined in section 4 above for modified versions, provided that you include in the combination all of the Invariant Sections of all of the original documents, unmodified, and list them all as Invariant Sections of your combined work in its license notice. 

	The combined work need only contain one copy of this License, and multiple identical Invariant Sections may be replaced with a single copy. If there are multiple Invariant Sections with the same name but different contents, make the title of each such section unique by adding at the end of it, in parentheses, the name of the original author or publisher of that section if known, or else a unique number. Make the same adjustment to the section titles in the list of Invariant Sections in the license notice of the combined work. 

	In the combination, you must combine any sections entitled "History" in the various original documents, forming one section entitled "History"; likewise combine any sections entitled "Acknowledgements", and any sections entitled "Dedications". You must delete all sections entitled "Endorsements." 

	\subsection{Collections of Documents}

	You may make a collection consisting of the Document and other documents released under this License, and replace the individual copies of this License in the various documents with a single copy that is included in the collection, provided that you follow the rules of this License for verbatim copying of each of the documents in all other respects. 

	You may extract a single document from such a collection, and distribute it individually under this License, provided you insert a copy of this License into the extracted document, and follow this License in all other respects regarding verbatim copying of that document. 

	\subsection{Aggregation with independant Works} 

	A compilation of the Document or its derivatives with other separate and independent documents or works, in or on a volume of a storage or distribution medium, does not as a whole count as a Modified Version of the Document, provided no compilation copyright is claimed for the compilation. Such a compilation is named an "aggregate", and this License does not apply to the other self-contained works thus compiled with the Document, on account of their being thus compiled, if they are not themselves derivative works of the Document. 

	If the Cover Text requirement of section 3 is applicable to these copies of the Document, then if the Document is less than one quarter of the entire aggregate, the Document's Cover Texts may be placed on covers that surround only the Document within the aggregate. Otherwise they must appear on covers around the whole aggregate. 

	\subsection{Translation}

Translation is considered a kind of modification, so you may distribute translations of the Document under the terms of the corresponding section about transformation. Replacing Invariant Sections with translations requires special permission from their copyright holders, but you may include translations of some or all Invariant Sections in addition to the original versions of these Invariant Sections. You may include a translation of this License provided that you also include the original English version of this License. In case of a disagreement between the translation and the original English version of this License, the original English version will prevail. 

	\subsection{Termination}

	You may not copy, modify, sublicense, or distribute the Document except as expressly provided for under this License. Any other attempt to copy, modify, sublicense or distribute the Document is void, and will automatically terminate your rights under this License. However, parties who have received copies, or rights, from you under this License will not have their licenses terminated so long as such parties remain in full compliance. 

	\subsection{Future revisions of this License}

	The Free Software Foundation may publish new, revised versions of the GNU Free Documentation License from time to time. Such new versions will be similar in spirit to the present version, but may differ in detail to address new problems or concerns. See \href{http://www.gnu.org/copyleft/}{{\color{blue} http://www.gnu.org/copyleft/}}. 

	Each version of the License is given a distinguishing version number. If the Document specifies that a particular numbered version of this License "or any later version" applies to it, you have the option of following the terms and conditions either of that specified version or of any later version that has been published (not as a draft) by the Free Software Foundation. If the Document does not specify a version number of this License, you may choose any version ever published (not as a draft) by the Free Software Foundation. 
	
	\begin{center}
	\color{ForestGreen}{{\Large \faTree} \textbf{Please consider the environment before printing}}
	\end{center}
	
	%to make section start on odd page
	\newpage
	\thispagestyle{empty}
	\mbox{}
	\section{Roadmap}
	This book has a simple progression rule that is: $1$ new A4 page by day since May 2001 on subjects that interest the supervisor of the \textit{Sciences.ch} distribution of the book \textit{Opera Magistris}. The following subjects below are already planned for a near of far future still with the same level of details and pedagogical approach in the proofs:
	\begin{itemize}
		\item Probabilites:
			\begin{itemize}
				\item Baysian conjugation for Normal and Binomial law
				\item Hidden Markov Chains
			\end{itemize}
		\item Statistics: 
			\begin{itemize}
				\item Mode and Median of statistical laws				
				\item Semi-variance	
				\item Partial and semi-partial correlation
				\item M-Estimators for localization and for dispersion
				\item Likelihood of censored data
				\item Jensen Inequality
				\item Normal Law Entropy
				\item Maximum likelihood Test
				\item Propension score
				\item Equivalence test
				\item Quasi-correlation matrix
				\item Factorial Analysis
				\item Hotelling T-Test
				\item Welch Test with Welch-Satterhwaite equation
				\item ANCOVA
				\item Wald-Wolfowitz Test (binary sequence)
				\item Levene-Wolfwitz Test\footnote{also named "turning point test" or "trend test"} (continuous up/down sequence)
				\item Odds Ratio and its confidence interval
				\item Risk Ratio and its confidence interval
				\item Ellipse of control
				\item Poisson Model for the average (2D) spatial distance
				\item Canonical Correlation
				\item Intraclass correlation coefficient (ICC)
				\item G-test of periodicity
				\item Gaussian and Student copula
				\item Hierarchical Fixed Factor ANOVA
				\item Square Latin ANOVA without replication
				\item Introduction to MANOVA
				\item Extreme Values Theorem
				\item Survey Theory
				\item Generalized Linear Models (Gauss, Poissson, Negative Binomial, Gamma)
				\item Logistic regression based on maximum likelihood
				\item PLS Regression (partial least squares)
				\item Logic regression
			\end{itemize}
		\item Sequences and Series:
			\begin{itemize}
				\item Properties of Fourier transforms	
				\item Discrete Fourier transform			
				\item Laplace Transform
				\item Z transform (common Z transforms, inverse common Z transforms)
				\item Converenge of Bessel Series
			\end{itemize}
		\item Differential Calculus
			\begin{itemize}
				\item Lebesgue Integral with numerical application in MATLAB™				
				\item Laplace Method
				\item Continuous and Discrete Convolution
			\end{itemize}		
		\item Functional Analysis: 
			\begin{itemize}
				\item Convexity and Concavity of a function
				\item Orthogonality of Hermite polynomial
			\end{itemize}
		\item Complex Analysis: 
			\begin{itemize}
				\item Residue Theorem for polynomial ratios
			\end{itemize}
		\item Topology: 
		\begin{itemize}
			\item Mahalanobis Distance
		\end{itemize}			
		\item Differential Geometry: 
			\begin{itemize}
				\item Normal coordinates
				\item Gauss curvature
				\item Isoperimetric plane theorem
			\end{itemize}	
		\item Mechanics: 
			\begin{itemize}
				\item Magnus effect
			\end{itemize}	
		\item Electrodynamics:
			\begin{itemize}		
				\item Lorentz oscillator model
				\item Electromagnetic Fields of a rotating shell of charge
			\end{itemize}
		\item Optical Wave:
			\begin{itemize}		
				\item Fresnel Diffraction
				\item Fraunhoffer Diffraction
				\item Optical Fiber basics
			\end{itemize} 
		\item Astronomy:
			\begin{itemize}	
				\item MacCullagh's formula
				\item Body flatness indirect calculation
				\item Synchronous locking of tidally evolving satellites		
			\end{itemize}		
		\item General Relativity:
			\begin{itemize}
				\item Real volume of an object in General Relativity
				\item Einstein radius derivation			
			\end{itemize}
		\item Cosmology:
			\begin{itemize}
				\item Derivation of Friedman equations from FRWL metric
			\end{itemize}
		\item Atomistic:
			\begin{itemize}
				\item Rayleigh diffusion (Rayleigh scattering)
				\item Neutron transport
				\item Parity, charge conjugation and time reversal (CPT)
				\item Photon "spin" (helicity) and polarity relation
				\item Bell inequalities
				\item Kennard inequality of Heisenberg incertitude
				\item Lamb shift detailed calculation
				\item Davisson-Germer Experiment
				\item EPR paradox formalism
			\end{itemize}
		\item Chemistry:
			\begin{itemize}
				\item Molecular Rotational Energy and Electron Transitions	
				\item Vibrational Energy of Molecules
				\item Vibrational plus Rotational Energy of Molecules
			\end{itemize}
		\item Numerical Methods: 
			\begin{itemize}
				\item Univariate optimization problem with substitution method					
				\item Acceptation/Rejection Sampling
				\item Gibbs Sampling				
				\item Outliers vs Influential values
				\item Cronbach coherence indicator
				\item Linear discriminant Analysis
				\item Quadratic discriminant Analysis
				\item Multidimensional scaling (MDS)
				\item Linear Mixture Model (LMM)
				\item Kernel Smoothing
				\item Mean Shift
				\item Factorial Analysis
				\item Correspondence Factorial Analysis
				\item GRG Generalized Reduced Gradient (GRG) optimization method
			\end{itemize}
		\item Quantum Computing: 
			\begin{itemize}
				\item No-cloning theorem
			\end{itemize}	
		\item Engineering:
			\begin{itemize}
				\item Box Domains
				\item Central Composite Design
				\item Center Face Cube Design
				\item Cox Survival Model (Cox Proportional Hazard Model)
				\item Modelization by Structural Equations				
				\item Accelerated life testing
				\item Microelectronics (npn/pnp jonctions, diodes, amplifiers)
				\item Telegraphe equation
				\item Kutta.Joukowski lift theorem
			\end{itemize} 
		\item Economy: 
			\begin{itemize}
				\item Continuous Yield rate
				\item Zero-Coupon curve rates
				\item Equivalence of an obligation rate for a treasure bond
				\item Spot rate and Forward rate
				\item Adjusting the beta of a portfolio with Futures
				\item Cox-Ingersoll Future/Forward price equality 
				\item Solution of Black \& Scholes ODE
				\item Black Model
				\item Macaulay Duration 
				\item Modified Duration
				\item Modified Internal Rate of Return (MIRR)
				\item Options Portfolio hedging
				\begin{itemize}
					\item Protective Put/Call
					\item Bull Spread/Call
					\item Bear Spread/Call
					\item Butterfly
					\item Straddle
					\item Strangle
					\item Collar
					\item Calendar spreads
					\item Portfolio allocation methods
					\begin{itemize}
						\item Optimal weighted portfolio for balanced risk 
						\item Optimal weighted portfolio for error tracking
						\item Optimal weighted Sharp's portfolio
						\item Optimal weighted portfolio with maximum diversification
						\item Optimal market-bench weighted Treynor-Black Portfolio
					\end{itemize}
				\end{itemize}
				\item Surplus at Risk (SVaR)
				\item Default Credit Risk (based on Standard \& Poor rating)
				\item VaR Equity Coverage
				\item Condition VaR loss (CVaR)
				\item Fokker-Planck equation
				\item ARCH-GARCH stochastic process
				\item Vector autoregressive models for multivariate time series
			\end{itemize}	
		\item Quantitative Management: 
			\begin{itemize}
				\item Gale-Shapley Algorithm
				\item Newsvendor problem
				\item Bullwhip Effect
				\item Condorcet paradox	
				\item Computerized Relative Allocation of Facilities Technique (CRAFT)			
				\item Real options
				\item Procedural Hierarchical Analysis
				\item Differed Capital in living case (life assurance)
				\item Modified Duration
				\item Death differed temporary (life assurance)
			\end{itemize}	
	\end{itemize}
	Remember that the \LaTeX{} sources of this book can be obtained actually depending on your donation on Patreon, Paypal or Tipee.
	
	As every robust product has a lifecycle. The lifecycle begins when a product is released and ends when it's no longer supported. Knowing key dates in this lifecycle helps you make informed decisions about when to upgrade. This book has the following lifecycle: a new major or minor version is published every 1st of month following the Gregorian Calendar and can be downloaded with by clicking on the following button (270MB PDF...):
	\begin{center}
		\href{http://www.sciences.ch/htmlfr/php/cliccount/click.php?id=317}{\includegraphics[scale=0.6]{img/books/download.jpg}}
	\end{center}
	or if this link would not work, a copy of the PDF is available on the Internet archive:
	\begin{center}
		\includegraphics[scale=0.1]{img/internet_archive.jpg}
	\end{center}
	\begin{center}
	\href{https://archive.org/details/OperaMagistris}{https://archive.org/details/OperaMagistris}
	\end{center}
	
	To quote this book:
	\begin{quote}
	\noindent @book\{OperaMagistris2013v3, \\
		  author =       \{Vincent Isoz and Léon Harmel\}, \\
		  title =        \{Opera Magistris - Elements of Applied Mathematics for Engineers\}, \\
		  year =         \{2014\}, \\
	      publisher=     \{Sciences.ch\}, \\
		  keywords =     \{science, physics, maths, engineering, finance, management\}, \\
		  isbn =          \{978239909327\},\\
	\}
	\end{quote}